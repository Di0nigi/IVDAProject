\documentclass{article}

\usepackage[english]{babel}
\usepackage[utf8]{inputenc}
\usepackage{amsmath,amssymb}
\usepackage{parskip}
\usepackage{graphicx}
\usepackage[numbers]{natbib} 
\usepackage{float} 
\usepackage{ai-usage-card}
% Margins
\usepackage[top=1cm, left=2cm, right=2cm, bottom=4.0cm]{geometry}
% Colour table cells
\usepackage[table]{xcolor}

% Get larger line spacing in table
\newcommand{\tablespace}{\\[1.25mm]}
\newcommand\Tstrut{\rule{0pt}{2.6ex}}         % = `top' strut
\newcommand\tstrut{\rule{0pt}{2.0ex}}         % = `top' strut
\newcommand\Bstrut{\rule[-0.9ex]{0pt}{0pt}}   % = `bottom' strut

%%%%%%%%%%%%%%%%%
%     Title     %
%%%%%%%%%%%%%%%%%
\title{{Group Assignment 1: Project Characterization \\ \small Interactive-Visual Data Analysis, Fall 2024}}
\author{\textbf{INSERT HERE: Dicypa, Full names and student numbers of your entire group}}
\begin{document}
\maketitle

%%%%%%%%%%%%%%%%%
%     Content     %
%%%%%%%%%%%%%%%%%
\section*{What}
We have set out to make an interactive surface for Grata Franzini's manhually collected ``Catalogue of Digital Editions''. The catalogue consists of a list of 350 
 works from digital humanities projects with a total of 52 attributes each (including an ID per item), obtained in tabular form as a CSV file from her GitHub repository  
 (\href{Link}{https://github.com/dig-Eds-cat/digEds_cat}), which is  periodically updated with new works and maintained for accuracy. 
 Of the 52 attributes, 25 are encoded as dummy variables, 15 as text (i.e. links, names, descriptions), 
 8 as categorical variables, and 4 as numerical variables., of which 3 are dates/years.
 Some of the dummy variables refer to inherent properties of the actual digital edition (susch as ``Account for Textual Variance'', which describes 
 if an edition takes into account textual variations between different versions of the considered work), while others refer to the presence technical properties
 (e.g. ``XML-TEI Transcription'', which indicates if the item is available as such). Categorical variables allow the editions to be more finely described, again by inherent attribute
 e.g. the historical period (in 8 categories, in this case even ordinal, though this doubles in part with the publishing century) or technical details (e.g. indicating the type of OCR used). 
 Works range From the antiquity to contemporary, with ``Middle Ages'', `` Long Nineteenth Century'' and `` Early Modern'' dominating. Predominantly, editions are
 in either English, Latin or German, whcih comprise more than half of the mostly manuscript based catalogue. Of the 18'252 cells, only 8 have missing values, concentrated in just three columns.
 6 Items are missing their HAndle-PID (a persistent identifier), one is missing information on the funding body and for one the catalogue does not specify wether the edition considers the `` value of witnesses''.
 The majority of items provide a XML-TEI transcription, although most refuse the ability to download this. Just over half of the editions are open access, with some more being partially open source.
The catalogue has a website already (\href{Link}{https://dig-ed-cat.acdh.oeaw.ac.at/editions.html}), but it is very barebones when it comes to visual exploration. Only a map of the funding institutions is provided (apart from the tabluar representation with categorical filtering). 

-still missing: what are we actually using
-Data wrangling: little to none, deal with the missing values.

\section*{Why}
\textbf{Note:} I talked to my friend with a histry master about this. Yes filters are useful and necessary, but apparently, if we are thinking about exploration,
the main thing for historians is not just an overview, but rather the ability to go down a rabbit hole. I.e. they are interested in connections betweeen the works/authors etc.
That may also be a way to incorporate LLM generation, more as a POC than as actual functionality, which would have to be provided by linking things to other databases such as \href{Monumenta Germanie Historica}{https://www.mgh.de/de}

\begin{itemize}
    \item Explore all data
    \item Discover relations (!!!)--Dependencies between items and similarities and differences
    \item Show Clusters
    \item Find interesting works
    \item Focus should be on Consumption: Discover, Enjoy
    \item Exploratory search: combining open-ended information-seeking behavior (exploration) with directed search and exploration support
\end{itemize}
\section*{How}
% YOUR RESPONSE HERE

\section*{Group Dynamics}
% YOUR RESPONSE HERE

%%%%%%%%%%%%%%%%%
\newpage
%     Example reference in IEEE style:     %
\bibliographystyle{IEEEtranN}
\bibliography{references}
\newpage
\makeAIUsageCard
%%%%%%%%%%%%%%%%%
\end{document}
