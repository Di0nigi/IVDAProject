\documentclass{article}

\usepackage[english]{babel}
\usepackage[utf8]{inputenc}
\usepackage{amsmath,amssymb}
\usepackage{parskip}
\usepackage{graphicx}
\usepackage[numbers]{natbib} 
\usepackage{float} 
\usepackage{ai-usage-card}
% Margins
\usepackage[top=1cm, left=2cm, right=2cm, bottom=4.0cm]{geometry}
% Colour table cells
\usepackage[table]{xcolor}



% Get larger line spacing in table
\newcommand{\tablespace}{\\[1.25mm]}
\newcommand\Tstrut{\rule{0pt}{2.6ex}}         % = `top' strut
\newcommand\tstrut{\rule{0pt}{2.0ex}}         % = `top' strut
\newcommand\Bstrut{\rule[-0.9ex]{0pt}{0pt}}   % = `bottom' strut

%%%%%%%%%%%%%%%%%
%     Title     %
%%%%%%%%%%%%%%%%%
\title{{Group Assignment 1: Project Characterization \\ \small Interactive-Visual Data Analysis, Fall 2024}}
\author{\textbf{Dicypa: Dionigi Rodriguez 24-755-688 \\ Cyril Smetanka 24-754-434 \\ Patrick Sproll 19-733-104}}
\begin{document}
\maketitle

%%%%%%%%%%%%%%%%%
%     Content     %
%%%%%%%%%%%%%%%%%
\section*{What}
We have set out to make an interactive surface for Grata Franzini's manually collected ``Catalogue of Digital Editions''. The catalogue consists of a list of 350
works from digital humanities projects with a total of 52 attributes each (including an ID per item), obtained in tabular form as a CSV file from her GitHub repository
(\href{https://github.com/dig-Eds-cat/digEds_cat}{Link}), which is  periodically updated with new works and maintained for accuracy. 
Of the 52 attributes, 25 are encoded as dummy variables, 15 as text (i.e. links, names, descriptions),
8 as categorical variables, and 4 as numerical variables., of which 3 are dates/years.
Some of the dummy variables refer to inherent properties of the actual digital edition (susch as ``Account for Textual Variance'', which describes
if an edition takes into account textual variations between different versions of the considered work), while others refer to the presence technical properties
(e.g. ``XML-TEI Transcription'', which indicates if the item is available as such). Categorical variables allow the editions to be more finely described, again by inherent attribute
e.g. the historical period (in 8 categories, in this case even ordinal, though this doubles in part with the publishing century) or technical details (e.g. indicating the type of OCR used).
Works range From the antiquity to contemporary, with ``Middle Ages'', `` Long Nineteenth Century'' and `` Early Modern'' dominating. Predominantly, editions are
in either English, Latin or German, whcih comprise more than half of the mostly manuscript based catalogue. Of the 18'252 cells, only 8 have missing values, concentrated in just three columns.
6 Items are missing their Handle-PID (a persistent identifier), one is missing information on the funding body and for one the catalogue does not specify wether the edition considers the `` value of witnesses''. Given the data was prepared by an experienced researcher,
we consider it both reliable (safe for an occasional dead link) and clean, requiring little to no preprocessing except the handling ot the missing values.
The majority of items provide a XML-TEI transcription, although most refuse the ability to download this. Just over half of the editions are open access, with some more being partially open source.
We imntent to augment the dataset with a number of LLM generated attributes in order to facilitate exploration and assessment of works by researchers through similarities and connections between items: 
Author school of thought, 5 keywords based on the contents of a works url, author and title, A categorical statement on authoritativeness, similarly a categorical statement for the renown of a work as well as a 
quick work description to provide some detail on the contents of a work dujring exploration. We intend to use mainly intrinsic attributes for the project, not the technical ones, as the target audience is
primarily interested in these aspects.
The catalogue has a website already (\href{https://dig-ed-cat.acdh.oeaw.ac.at/editions.html}{Link}), but it is very barebones when it
 comes to visual exploration. Only a map of the funding institutions is provided (apart from the tabluar representation with categorical filtering).

-go over this with the slides on the side.

\section*{Why}
The goal according to the problem statement is to let users (primarily Historians, Researches and Digital Library users) explore the contents of the catalogue.
Naturally, the catalogue also needs to enable the users to do directed search, in case they already know what they are looking for. A such, the task falls squarely into the data
 exploration and Interactive relation discovery research streams. We are considering a sub tool that would enable users to tag their own connections between works, 
 which would touch upon interactive data labeling. Correspondingly we have settled on the following tasks to be covered:

%actual tasks selected
\begin{itemize}
    \item Overview the data -- Given the large amount and different types of works, users need to get a grasp 
    of what they are able to find in the catalogue and if there may be something that fits their need and interest, 
    focused on intrisic attributes (Author, LLM generated keywords, School of thought Period, Time, Account of textual variance,Value of witnesses, etc.), not technical ones. 
    As Part of the overview we will provide a clustering of similar works.
    \item Identify relationships -- Historians like to go into a rabbit hole from one item to the next. We want to facilitate this by guiding exploration among relationships between works.
    (author school of thought, author, time, period, keywords, description, institution). We will use clustering to find similar works, as well as LLM generated keywords.
    \item Filter by attributes -- In order to reduce noise and narrow down the search space, users need to be able to filter by all attributes, technical and non technical. This will be done using traditional filters.
    \item Judge reliability -- In order to make an informed decision on what works to use, users need to be able to assess the quality of the data. We will provide LLM generated attributes such as renown and 
    authoritativeness along side the values contained in the catalogue (sponsor/founding body, repository of source materials, citations, audience, philiological statement, value of witnesses, ). We will provide an (likely) 
    ML based overall score as well.
    \item Tag relationships -- In order to record their own train of thought and help organize their research, users need to be able to tag relationships between works. We will likely provide a user defineable attribute graph for this.
\end{itemize}


- go through these points with the slides on the side again


\section*{How}
%1
\begin{itemize}
    \item Gain overview of the data; timeline with ages colored, block of title author
          and description of a work, histogram of languages, density map based on source
          material location, predefined version of the keyword graph
    \item Identify relationships; Graph (used defined/premade), Map?, Cluster plot
    \item Filter by attributes; historgams/bar chart like in the lecture, search
          function, glkobal filter model
    \item judge reliabilty; User generated score represented as bar
    \item Tag relationships; Graph (used defined) with field text
\end{itemize}

Always encode with colorblind colors.

%2
sketched

%3
zooming on timeline/cluster plot, lasso selection in graphs, clicking for
details on demand, typing fro search, drag drop for filter and user defined
graph if possible, sliders,

%4 ML: 
kmeans/Knn(supervised) for clustering, soemthing for the reliabilty score...
dont forget the reference to some tutorial etc

% YOUR RESPONSE HERE

\section*{Group Dynamics}

\begin{itemize}
    \item Data shaper and data steward are not really necesasry, the data is there and
          mostly clean
    \item Dionigi is ML/AI Engineer / Data engineer-> mostly backend work
    \item Cyril will mostly be the generalist -> Frontend / documentation
    \item Patrick will be the technical analyst -> Frontend / Documentation
    \item No research scientist present
\end{itemize}

% YOUR RESPONSE HERE

%%%%%%%%%%%%%%%%%
\newpage
%     Example reference in IEEE style:     %
\bibliographystyle{IEEEtranN}
\bibliography{references}
\newpage
\makeAIUsageCard
%%%%%%%%%%%%%%%%%
\end{document}
